\documentclass{article}

\usepackage{graphicx}
\usepackage{amsmath} 

\usepackage{xepersian}
\settextfont{B Nazanin}


\begin{document}



در این بخش با انواع  \lr{OFDM} نوری آشنا خواهیم شد.
\section{\lr{ACO-OFDM}}


در \lr{ACO-OFDM}       \LTRfootnote{Asymmetrically Clipped Optical}
تنها زیر حامل های با اندیس فرد اطلاعات را انتقال میدهند. در حالی که زیر حامل های با اندیس زوج یک بایاس را ایجاد میکنند که باعث میشوند  سیگنال \lr{OFDM} ارسال شده بتواند غیر منفی شود.\cite{2}\ 
\begin{figure}[h!]
\begin{center}
\includegraphics[width=9cm, height=4cm]{aco.PNG}
\end{center}
\caption{  سیستم \lr{ACO-OFDM}}
\label{acomodel}

\end{figure}
\subsection{فرستنده و گیرنده}

 شکل \ref{acomodel} یک سیستم \lr{ACO-OFDM} را نشان میدهد. سیگنال ورودی به \lr{,\textbf{X},IFFT} تنها از اندیس های فرد تشکیل شده و اندیس های زوج برابر صفر قرار گرفته اند.\lr{\textbf{X}$ = [0,X_{1},0,X_{3},....,X_{N-1}]$} ‌همچنین وکتور  \lr{\textbf{X}} دارای تقارن هرمیتی  \LTRfootnote{hermitian symmetry} میباشد چرا که میخواهیم وکتور خروجی  \lr{,\textbf{x},IFFT}  حقیقی باشد.\cite{1} سیگنال خروجی  \lr{,\textbf{x},}  حقیقی میباشد و دارای ویژگی ضد تقارن \LTRfootnote{anti symmetry} است  همانطور که در رابطه زیر میبینیم.\cite{6}

\begin{equation}
x_{k} = -x_{k+\frac{N}{2}}    \hspace{8pt}\textrm{\lr{for}}  \hspace{4pt}  0 < k < \frac{N}{2} .
\label{acoanti}
\end{equation}

فرستنده \lr{ACO-OFDM} خیلی شبیه به \lr{DCO-OFDM} است. ابتدا \lr{x} سریال میشود و سپس \lr{CP} \LTRfootnote{Cyclic Perfix} را انجام میدهیم.  \lr{x}  به آنالوگ تبدل میشود و خروجی از یک فیلتر پایین گذر ایده آل عبور میکند و نتیجه$ x(t)‌$ میشود. به دلیل این که نمونه های منفی نمیتوانند ارسال شوند چون سیستم \lr{IM/DD} است. در صفر قطع میشود و سیگنال \lr{ $x_{ACO}(t)$, ACO-OFDM}حاصل میشود. بخاطر خاصیت ضد تقارن این سیگنال که در رابطه \ref{acoanti}  مشاهده شد با بریدن و قطع کردن سیگنال در صفر هیچ اطلاعاتی از بین نمی رود. هم چنین با بریدن و قطع کردن سیگنال یک نویز جدید مانند تداخل روی زیرحامل های زوج به وجود می آید.\cite{4} سپس  \lr{ $x_{ACO}(t)$} به عنوان ورودی به فرستنده نوری ایده آل داده می شود و ساختار گیرنده هم دقیقا مانند گیرنده \lr{DCO-OFDM}  است با این تفاوت که در \lr{ACO-OFDM} ‌ تنها زیر حامل های فرد دمودله میشوند چرا که تنها این زیر حامل ها اطلاعات را انتقال می دهند.

\subsection{محاسبه احتمال خطا}
در این قسمت احتمال خطا \lr{ACO-OFDM} مورد بررسی قرار میگیرد و فرمول تئوری بدست آمده از مقاله فلان همراه با شبیه سازی سیستم در یک نمودار نشان داده میشوند. فرمول احتمال خطای \lr{ACO-OFDM} در زیر آمده است.

\begin{equation}
BER_{ACO} = \frac{2(\sqrt{M}-1)}{\sqrt{M}\times \log_{2}\sqrt{M}} \times Q(\sqrt{\frac{3\times SNR}{M - 1}})
\label{acober}
\end{equation}


در نمودار زیر نیز هم نمودار های تئوری و هم نمودارهای حاصل از شبیه سازی رسم شده اند.

\begin{figure}[h!]
\begin{center}
\includegraphics[width=8cm, height=6cm]{acober.PNG}
\end{center}
\caption{  سیستم \lr{ACO-OFDM}}
\label{acomodel}

\end{figure}

\section{\lr{DCO-OFDM}}
روش دیگری که در \lr{Optical-OFDM}\ در سیستم های مخابرات نوری استفاده می شود روش \lr{DCO- OFDM}       \LTRfootnote{DC biased optical OFDM}
می باشد.\lr{OFDM}\ معمولی در انتها یک سیگنال مختلط ایجاد می‌کند اما چون در سیستم های مخابراتی نوری از \lr{LED} \LTRfootnote{Light Emitting Diode} استفاده می شود و \lr{LED} نیز فقط قادر به مدوله کردن شدت نور می باشد، بنابراین سیگنالی که ایجاد می شود باید حقیقی و مثبت باشد. 

\begin{figure}[h!]
\begin{center}
\includegraphics[width=9cm, height=5cm]{dco.PNG}
\end{center}
\caption{  سیستم \lr{DCO-OFDM}}
\label{dcomodel}

\end{figure}
\subsection{فرستنده و گیرنده}
خروجی سیستم \lr{IFFT}\ مطابق با رابطه‌ی زیر به دست می‌آید:
\begin{equation}
x_{m} = \frac{1}{N}\sum_{m=0}^{N-1}X_{k}e^{j \frac{2 \pi km}{N}}    \hspace{8pt}\textrm{\lr{for}}  \hspace{4pt}  0 \leq m \leq N-1 .
\label{dcoanti}
\end{equation}

در تکنیک \lr{DCO-OFDM}\ بردار \lr{X}\ را به گونه ای در نظر می‌گیرند که خروجی رابطه‌ی  \ref{dcoanti} یک سیگنال حقیقی بشود.
\begin{equation}
X=[X_{1},X_{2},X_{3},...,X_{N-1}]
\label{ofdmseq}
\end{equation}
\begin{equation}
X_{DCO}=[0,X_{1},X_{2},X_{3},...,X_{N-1},0,X_{N-1}^{*},X_{N-2}^{*},X_{N-3}^{*},...,X_{1}^{*}]
\label{dcoseq}
\end{equation}

در این صورت طول \lr{X}\ برابر با \lr{2N} می‌شود و رابطه‌ \lr{IFFT}\ برای \lr{X}\ را به صورت زیر می‌توان نوشت:
\begin{equation}
x_{m} = \frac{1}{N}\sum_{m=0}^{2N-1}X_{k}e^{j \frac{2 \pi km}{N}}    \hspace{8pt}\textrm{\lr{for}}  \hspace{4pt}  0 \leq m \leq 2N-1 .
\label{dcoifft}
\end{equation}
تا اینجا با استفاده از تکرار کردن اطلاعات و از دست دادن بهره‌ی طیفی سیگنال حقیقی به دست آمده است.اما در مرحله‌ی بعدی طبق شکل \ref{dcomodel}\ اضافه کردن  \lr{Cyclic Prefix}\ می‌باشد. در این بلوک اگر طول رشته‌ی ورودی \lr{L}\ باشد، به تعداد \lr{L/4}\ از انتهای رشته را در ابتدای رشته‌ی ورودی کپی می‌کنیم. این کار باعث می‌شود که گیرنده دچار \lr{ISI} \LTRfootnote{Inter Symbol interference}\ نشود.

در مرحله‌ی بعدی که اصلی ترین قسمت روش \lr{DCO-OFDM} می‌باشد، برای ایجاد یک سیگنال حقیقی و مثبت باید مقادیر منفی سیگنال را به مقادیر مثبت تبدیل کرد. بنابراین یک مقدار \lr{DC} به سیگنال افزوده می‌شود. این مقدار بایاس طبق شکل \ref{dcomodel}\ در حالت پیوسته به سیگنال افزوده می‌شود.
\begin{equation}
x_{DCO}(t)=x(t)+B_{DC} \hspace{8pt}
\label{dcox}
\end{equation}

بعد از این مراحل سیگنال دیگر تغییری پیدا نمی‌کند و درگیرنده ابتدا با فرض سنکرون بودن گیرنده و فرستنده قسمت \lr{Cyclic Prefix}\ حذف می‌شود.پس از آن از حوزه‌ی زمان با استفاده از ماژول \lr{FFT}\ به حوزه‌فرکانس بازمی گردد.

\lr{Single Tap Equalizer} نیز برای حذف اثرات کانال می‌باشد. دراین قسمت در ضمن می توان اثراتی هم چون \lr{Multi-Path fading} را نیز کم کرد و پس از حذف اثرات کانال مطابق با آن مدلاسیونی که در فرستنده استفاده شده است، دمدولاسیون انجام می‌شود و سیگنال اطلاعات به دست می‌آید.  

یکی از معایب این روش مصرف انرژی زیاد می‌باشد و از لحاظ انرژی خیلی بهینه نمی باشد، زیرا که در فرستنده با افزودن بایاس به سیگنال انرژی سیگنال زیاد می‌شود و هم چنین با افزودن سیگنال بایاس به سیگنال اصلی احتمال اشباع شدن بیشتر می‌شود.


\subsection{شبیه سازی}
در این قسمت شبیه سازی این روش در متلب انجام شده است و نمودار \lr{Bit Error Rate}\ را به دست آورده‌ایم. در ابتدا سیگنال تصادفی‌ای را به عنوان 

\setLTRbibitems
\begin{thebibliography}{9}
\resetlatinfont
\bibitem{2}
J. Armstrong and B. J. C. Schmidt, “Comparison of asymmetrically
clipped optical OFDM and DC-biased optical OFDM in AWGN,”
IEEE Commun. Lett., vol. 12, pp. 343–345, 2008.
\bibitem{1}
H. Elgala, R. Mesleh, and H. Haas, “Practical considerations for indoor wireless optical system implementation using OFDM,” in Pro.
ConTEL, Zagreb, Croatia, 2009, pp. 25–30.
\bibitem{4}
J. Armstrong and A. J. Lowery, “Power efficient optical OFDM,” Electron. Lett., vol. 42, pp. 370–372, 2006.
\bibitem{6}
K. Asadzadeh, A. Dabbo, and S. Hranilovic, “Receiver design for
asymmetrically clipped optical OFDM,” in Proc. IEEE GLOBECOM
OWC Workshop, Houston, TX, USA, 2011.
\end{thebibliography}


\end{document}



\LTRfootnote{AC}
\textbf{X}
\cite{6}